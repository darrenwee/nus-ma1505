\documentclass[10pt,landscape]{article}
\usepackage{multicol}
\usepackage{calc}
\usepackage{ifthen}
\usepackage[landscape]{geometry}
\usepackage{hyperref}
\usepackage{amssymb,amsmath}

% To make this come out properly in landscape mode, do one of the following
% 1.
%  pdflatex latexsheet.tex
%
% 2.
%  latex latexsheet.tex
%  dvips -P pdf  -t landscape latexsheet.dvi
%  ps2pdf latexsheet.ps



% This sets page margins to .5 inch if using letter paper, and to 1cm
% if using A4 paper. (This probably isn't strictly necessary.)
% If using another size paper, use default 1cm margins.
\ifthenelse{\lengthtest { \paperwidth = 11in}}
	{ \geometry{top=.5in,left=.5in,right=.5in,bottom=.5in} }
	{\ifthenelse{ \lengthtest{ \paperwidth = 297mm}}
		{\geometry{top=0.7cm,left=0.7cm,right=0.7cm,bottom=0.7cm} }
		{\geometry{top=0.7cm,left=0.7cm,right=0.7cm,bottom=0.7cm} }
	}

% Turn off header and footer
\pagestyle{empty}
 

% Redefine section commands to use less space
\makeatletter
\renewcommand{\section}{\@startsection{section}{1}{0mm}%
                                {-1ex plus -.5ex minus -.2ex}%
                                {0.5ex plus .2ex}%x
                                {\normalfont\large\bfseries}}
\renewcommand{\subsection}{\@startsection{subsection}{2}{0mm}%
                                {-1explus -.5ex minus -.2ex}%
                                {0.5ex plus .2ex}%
                                {\normalfont\normalsize\bfseries}}
\renewcommand{\subsubsection}{\@startsection{subsubsection}{3}{0mm}%
                                {-1ex plus -.5ex minus -.2ex}%
                                {1ex plus .2ex}%
                                {\normalfont\small\bfseries}}
\makeatother

% Define BibTeX command
\def\BibTeX{{\rm B\kern-.05em{\sc i\kern-.025em b}\kern-.08em
    T\kern-.1667em\lower.7ex\hbox{E}\kern-.125emX}}

% Don't print section numbers
\setcounter{secnumdepth}{0}


\setlength{\parindent}{0pt}
\setlength{\parskip}{0pt plus 0.5ex}


% -----------------------------------------------------------------------

\begin{document}

\raggedright
\footnotesize
\begin{multicols}{3}


% multicol parameters
% These lengths are set only within the two main columns
%\setlength{\columnseprule}{0.25pt}
\setlength{\premulticols}{1pt}
\setlength{\postmulticols}{1pt}
\setlength{\multicolsep}{1pt}
\setlength{\columnsep}{2pt}

\begin{center}
     \Large{\textbf{$\mathbb{MA}1505$ Cheat Sheet}} \\
\end{center}

\section{Functions}
\begin{tabular}{@{}ll@{}}
$(f \pm g)(x)=f(x)\pm g(x)$ & $\lim_{x\to a}(f\pm g)(x) = L \pm L'$ \\
$(fg)(x) = f(x)g(x)$  & $\lim_{x\to a}(fg)(x) = LL'$ \\
$(f/g)(x) = f(x)/g(x)$ & $\lim_{x\to a}\frac{f}{g}(x) = \frac{L}{L'}$ \\
 & $\lim_{x\to a} kf(x) = kL, k\in \mathbb{R}$\\
\end{tabular}
All polynomials are continuous at every point in $\mathbb{R}$.
All rational functions $\frac{p(x)}{q(x)}$ where $p$ and $q$ are polynomials are continuous at every point such that $q(x) \ne 0$.

Composition is given by $\circ$, e.g. $(f\circ g)(x) = f(g(x))$

\subsection{Differentiation}
\begin{tabular}{@{}ll@{}}
    Product Rule    & $(fg)'(x) = f'(x)g(x) + f(x)g'(x)$\\
    Quotient Rule   & $\Big(\frac{f}{g}\Big)'(x) = \frac{f'(x)g(x)-f(x)g'(x)}{g^2(x)}$ \\
    Chain Rule      & $(f\circ g)'(x) = f'(g(x))g'(x) = (f'\circ g)(x)g'(x)$ \\
                    & $\frac{dy}{dt} = \frac{dy}{dx}\times\frac{dx}{dt}$\\
\end{tabular}

\subsubsection{Maxima and Minima}
A function $f$ has a local/relative maximum value at a point $c$ in its domain if $f(x) \leq f(c)$ for all $x$ in the neighborhood of $c$.

The function has an absolute maximum value at $c$ if $f(x) \leq f(c)$ for all $x$ in the domain.

Reverse signs for minimum.

\subsubsection{Extreme and Critical Points}
Points where $f$ can have an extreme value are: interior points where $f'(x)=0$, interior points where $f'(x)$ does not exist and end points of the domain of $f$.

An interior point of the domain of a function $f$ where $f'$ is zero or does not exist is a \emph{critical point} of $f$.

\subsubsection{Increasing or Decreasing Functions}
$f$ is increasing on an interval $I$ when $f'(x) > 0$ for all $x\in I$.

$f$ is decreasing on an interval $I$ when $f'(x) < 0$ for all $x\in I$.

\subsubsection{Concavity}
The graph of $y=f(x)$ is concave down on any interval where $y''<0$ and concave up on any interval where $y''>0$.

A point $c$ is a point of inflection of the function $f$ if $f$ is continuous at $c$ and there is an open interval containing $c$ such that the graph of $f$ changes from concave up (or down) to concave down (or up).
The function need not be differentiable at $c$.

\subsubsection{Derivative Tests for Maxima and Minima}
\textbf{First derivative test:}
Suppose $c\in(a,b)$ is a critical point of $f$.

If $f'(x) > 0$ for $x \in (a,c)$ and $f'(x) < 0$ for $x\in (c,b)$ then $f(c)$ is a local maximum.

If $f'(x) < 0$ for $x \in (a,c)$ and $f'(x) > 0$ for $x\in (c,b)$ then $f(c)$ is a local minimum.

\textbf{Second derivative test:}
If $f'(c) = 0$ and $f''(c) < 0$ then $f$ has a local maximum at $x=c$.

If $f'(c) = 0$ and $f''(c) > 0$ then $f$ has a local minimum at $x=c$.

\subsection{Indeterminacy and L'Hopital's Rule}
If the functions $f$ and $g$ are continuous at $x=a$ but $f(a)=g(a)=0$, then the limit $\lim_{x\to a}\frac{f(x)}{g(x)}$ cannot be directly evaluated.

Suppose that $f$ and $g$ are differentiable in a neighborhood of $a$, $f(a) = g(a) = 0$ and $g'(x) \ne 0$ except possibly at $a$.
$$\lim_{x\to a} \frac{f(x)}{g(x)} = \lim_{x\to a} \frac{f'(x)}{g'(x)}$$

\section{Integration}
If $f$ is continuous on $[a,b]$, then
$$F(x) = \int_a^x f(t) dt$$
$$\frac{d}{dx} F(x) = \frac{d}{dx}\int_a^x f(t) dt = f(x)$$

Integration by parts.

$$\int u(x)v'(x) dx = u(x)v(x) - \int u'(x)v(x) dx$$

Area between two curves $f_2(x)$ and $f_1(x)$ where $f_1(x) \leq f_2(x)$ in $[a,b]$
$$\int_a^b f_2(x) - f_1(x) dx$$

\subsection{Partial Fraction Decomposition}
$$\frac{px+q}{(ax+b)(cx+d)} = \frac{A}{ax+b} + \frac{B}{cx+d}$$
$$\frac{px^2+qx+r}{(ax+b)(cx+d)^2} = \frac{A}{ax+b} + \frac{B}{cx+d} + \frac{C}{(cx+d)^2}$$
$$\frac{px^2+qx+r}{(ax+b)(x^2+c^2)} = \frac{A}{ax+b} + \frac{Bx+C}{x^2+c^2}$$

% \subsection{Integrals}
% \begin{tabular}{@{}rlr@{}}
%     $f(x)$              & $\int f(x) dx$                    & \\
%     $\frac{1}{x^2+a^2}$ & $\frac{1}{a}\arctan{\frac{x}{a}}$ & \\
% \end{tabular}

\section{Series}
\subsection{Arithmetic Series}
$$\sum_1^n a_n = \frac{n}{2}(a_1 + a_n)$$

\subsection{Geometric Series}
$$\sum_1^n ar^{n-1} = a\frac{1-r^n}{1-r}$$

If $|r| < 1$ then as $n\to\infty$,
$$\sum_1^n ar^{n-1} \to \frac{a}{1-r}$$

\subsection{Ratio Test}
For a series $\sum a_n$, let

$$\lim_{n\to\infty} \Big| \frac{a_{n+1}}{a_n}\Big| = \rho$$

Series is convergent if $\rho < 1$, divergent if $\rho > 1$ and no conclusion reached if $\rho = 1$.

\subsection{Power Series}
A power series has the form
$$\sum_{n=0}^\infty c_nx^n = c_0 + c_1x + c_2x^2 + ... + c_nx^n + ...$$

If the series is centered about $x=a$,
$$\sum_{n=0}^\infty c_n(x-a)^n = c_0 + c_1(x-a) + c_2(x-a)^2 + ... + c_n(x-a)^n + ...$$

\subsubsection{Standard Series}
$$(1+x)^r = 1+rx+\frac{r(r-1)}{2!}x^2 + ... + \frac{r(r-1)...(r-n+1)}{n!}x^n$$
$$\frac{1}{1-x} = 1 + x + x^2 + x^3 + ... + x^n + ... \text{for } |x| < 1$$
$$e^x = 1+x+\frac{x^2}{2!}+\frac{x^3}{3!} + ... + \frac{x^n}{n!} + ...$$
$$\sin{x} = x - \frac{x^3}{3!} + \frac{x^5}{5!} - ... + \frac{(-1)^nx^{2n+1}}{(2n+1)!} + ...$$
$$\cos{x} = 1-\frac{x^2}{2!} + \frac{x^4}{4!} - ... + \frac{(-1)^nx^{2n}}{(2r)!}$$
$$\ln{(1+x)} = x -\frac{x^2}{2} + \frac{x^3}{3} - ... + \frac{(-1)^{n+1}x^n}{n} + ...$$

\subsection{Radius of Convergence}
$$\lim_{n\to\infty} \Big| \frac{u_{n+1}}{u_n} \Big| < 1$$

\subsection{Taylor Series}
$$\sum_{k=0}^\infty \frac{f^{(k)}(a)}{k!}(x-a)^k = f(a) + f'(a)(x-a) + ... + \frac{f^{(n)}(a)}{n!}(x-a)^n + ... $$

\subsubsection{Taylor's Theorem}
The $n$th order Taylor polynomial of $f$ at $a$ is given by
$$P_n(x) = \sum_{k=0}^n \frac{f^{(k)}(a)}{k!}(x-a)^k = f(a) + f'(a)(x-a) + ... + \frac{f^{(n)}(a)}{n!}(x-a)^n$$

Then $f(x) = P_n(x) + R_n(x)$ where
$$R_n(x) = \frac{f^{(n+1)}(c)}{(n+1)!}(x-a)^{n+1}$$
for some $c$ between $a$ and $x$, where $R_n(x)$ is the remainder of order $n$ or the error term for the approximation of $f(x)$ by $P_n(x)$.

\section{Three Dimensional Space $\mathbb{R}^3$}
\subsection{Dot Product}
$$\vec{v_1} \cdot \vec{v_2} =
    \begin{pmatrix}
        x_1 \\
        y_1 \\
        z_1 \\
    \end{pmatrix}\cdot
    \begin{pmatrix}
        x_2 \\
        y_2 \\
        z_2 \\
    \end{pmatrix}
    = x_1x_2 + y_1y_2 + z_1z_2
    = |\vec{v_1}|\: |\vec{v_2}| \cos{\theta} $$

\subsection{Unit Vector}
For some vector $\mathbf{u}$, its unit vector $\mathbf{\hat{u}} = \frac{1}{|\mathbf{u}|}\mathbf{u}$


\subsection{Cross Product}
$$\vec{v_1} \times \vec{v_2} =
    \begin{pmatrix}
        x_1 \\
        y_1 \\
        z_1 \\
    \end{pmatrix}
    \times
    \begin{pmatrix}
        x_2 \\
        y_2 \\
        z_2 \\
    \end{pmatrix}
    =\begin{pmatrix}
        y_1z_2 - z_1y_2\\
        -(x_1z_2 - z_1x_2)\\
        x_1y_2 - y_1x_2\\
    \end{pmatrix}
    = |\vec{v_1}|\: |\vec{v_2}| \sin{\theta}$$

The distance $\text{dist}$ from a point $P(x_0,y_0,z_0)$ to a plane $\Pi:ax+by+cz=d$ is given by
$$\text{dist }=\frac{|ax_0+by_0+cz_0-d|}{\sqrt{a^2+b^2+c^2}}=\text{proj}_{\mathbf{n}}\vec{OP}$$

\subsection{Space Curves}
For some curve with the vector equation $\mathbf{r}(t) = f(t)\mathbf{i} + g(t)\mathbf{j} + h(t)\mathbf{k}$,

its arc length (if the curve is traversed once)
$$L = \int_a^b \sqrt{(f'(t))^2 + (g'(t))^2 + (h'(t))^2} dt = \int_a^b |\mathbf{r}'(t)|$$

\section{Fourier Series}
A function is said to be odd if $-f(x) = f(-x)$ and even if $f(x) = f(-x)$. Examples are $\sin{x}$ for the former and $\cos{x}$ for the latter.

A periodic function of period $T$ can be represented by a Fourier series $f(x)$.
$$f(x) = a_0 + \sum_{n=1}^\infty (a_n\cos{nx}+b_n\sin{nx})$$

Let $2L=T$.
$$a_0 = \frac{1}{2L} \int_{-L}^{L} f(x) dx$$

For the $m$th term where $m\in\mathbb{Z}^+$,
$$a_m = \frac{1}{L} \int_{-L}^{L} f(x) \cos{\frac{\pi mx}{L}} dx$$
$$b_m = \frac{1}{L} \int_{-L}^{L} f(x) \sin{\frac{\pi mx}{L}} dx$$

\textbf{If the function is even, we only need to consider cosine terms. Similarly, if the function is odd, we only need to consider sine terms.}

\section{Multivariate Functions}
\subsection{Partial Derivatives}
Let $z = f(x,y)$ be a function of two variables.

The \emph{partial derivative} of a function $f(x,y)$ w.r.t. $x$ is denoted by $f_x(x,y)$ or $\frac{\partial f}{\partial x}$ where the $y$ term is taken as a constant.

$$f_{xx} = (f_x)_x = \frac{\partial^2f}{\partial x^2} \text{ and } f_{xy} = (f_x)_y = \frac{\partial^2f}{\partial x\partial y}$$
$$f_{yy} = (f_y)_y = \frac{\partial^2f}{\partial y^2} \text{ and } f_{yx} = (f_y)_x = \frac{\partial^2f}{\partial x\partial y}$$

For most functions in practice, $f_{xy}(a,b) = f_{yx}(a,b)$.

\subsection{Chained Derivatives}
Suppose $z=f(x,y,z)$ where $x=x(t)$, $y=y(t)$ and $z=z(t)$. Thus, $z = f(x(t),y(t),z(t))$.
$$ \frac{dz}{dt} = \frac{\partial f}{\partial x}\frac{dx}{dt} + \frac{\partial f}{\partial y}\frac{dy}{dt} +\frac{\partial f}{\partial z}\frac{dz}{dt}$$

Suppose $w=f(x,y,z)$ and $x=x(s,t)$, $y=y(s,t)$, and $z=z(s,t)$, giving $w=f(x(s,t),y(s,t),z(s,t))$
$$ \frac{\partial w}{\partial s} = \frac{\partial f}{\partial x}\frac{\partial x}{\partial s} + \frac{\partial f}{\partial y}\frac{\partial y}{\partial s} + \frac{\partial f}{\partial z}\frac{\partial z}{\partial s}$$

$$\frac{\partial w}{\partial t} = \frac{\partial f}{\partial x}\frac{\partial x}{\partial t} + \frac{\partial f}{\partial y}\frac{\partial y}{\partial t} + \frac{\partial f}{\partial z}\frac{\partial z}{\partial t}$$

\subsection{Directional Derivatives}
Note that $D_{\mathbf{i}} f(a,b) = f_x(a,b)$ and $D_{\mathbf{j}}f(a,b) = f_y(a,b)$ for the standard unit vectors of the $x$ and $y$ direction.

For some \underline{unit vector} $\mathbf{\hat{u}} = u_1\mathbf{i} + u_2\mathbf{j}$,
$$D_{\mathbf{u}}f(a,b) = f_x(a,b)\cdot u_1 + f_y(a,b)\cdot u_2 = \nabla f(a,b)\cdot \hat{\mathbf{u}}$$

The directional derivative $D_{\mathbf{u}}f(a,b)$ measures the change in the value $df$ of a function $f$ when moved a distance $dt$ from the point $(a,b)$ in the direction of the vector $\mathbf{u}$, where $df = D_{\mathbf{u}}f(a,b)\cdot dt$.

\subsection{Gradient Vector}
The gradient vector $\nabla f$ is given by
$$\nabla f = f_x\mathbf{i} + f_y\mathbf{j}$$
$$D_{\mathbf{u}}f(a,b) = \nabla f(a,b)\cdot \mathbf{u} = |\nabla f(a,b)|\cos{\theta}$$

The function $f$ increases most rapidly in the direction $\nabla f(a,b)$ and decreases most rapidly in the direction $-\nabla f(a,b)$.

\subsection{Maxima and Minima}
$f(x,y)$ has a local maximum at $(a,b)$ if $f(x,y) \leq f(a,b)$ for all points $(x,y)$ near $(a,b)$

$f(x,y)$ has a local minimum at $(a,b)$ if $f(x,y) \geq f(a,b)$ for all points $(x,y)$ near $(a,b)$

A function $f$ may have a local maximum or minimum at $(a,b)$ if: $f_x(a,b)=0$ and $f_y(a,b)=0$ \underline{or} $f_x(a,b)$ or $f_y(a,b)$ is not defined.
A point that satisfies either condition is known as a \emph{critical point}.

Suppose that $(a,b)$ is a critical point of $f(x,y)$. Let us define $D$ as
$$D(a,b) = f_{xx}(a,b)f_{yy}(a,b)-[f_{xy}(a,b)]^2$$

\begin{tabular}{@{}rl@{}}
    $D>0$ and $f_{xx}(a,b)>0$   & relative minimum at $(a,b)$\\
    $D>0$ and $f_{xx}(a,b)<0$   & relative maximum at $(a,b)$\\
    $D<0$                       & saddle point at $(a,b)$\\
    $D=0$                       & no conclusion reached\\
\end{tabular}

\subsubsection{Saddle Point}
At a point $(a,b)$ of $f$ where $f_x(a,b) = 0$ and $f_y(a,b) = 0$, the point $(a,b)$ is known as a \emph{saddle point} of $f$ if there are some directions along which $f$ has a local maximum at $(a,b)$ and some directions along $f$ which has a local minimum at $(a,b)$.

\subsection{Lagrange Multiplier}
Suppose a function $f(x,y)$ subject to the constraint $g(x,y)$.
$$F(x,y,\lambda) = f(x,y) - \lambda g(x,y)$$

Solve for $F_x = 0$, $F_y = 0$ and $F_{\lambda} = 0$ to solve for $\lambda$.

\section{Multiple Integrals}
For $R = R_1 \cup R_2$ where $R_1$ and $R_2$ do not overlap except maybe at their boundary,
$$\iint_R f(x,y) dA = \iint_{R_1}f(x,y) dA + \iint_{R_2}f(x,y) dA$$

Suppose a rectangular region $R$ in the $xy$-plane where $a\leq x\leq b$ and $c\leq y\leq d$, then
$$\iint_R f(x,y) dA = \int_c^d\int_a^b f(x,y)\:dx\:dy = \int_a^b\int_c^d f(x,y)\:dy\:dx$$

\subsection{Type A Regions}
Bottom and top boundaries are curves given by $y=g_1(x)$ and $y=g_2(x)$ respectively, while left and right boundaries are $x=a$ and $x=b$ respectively.
$$R: g_1(x) \leq y \leq g_2(x),\:a\leq x\leq b$$
$$\iint_R f(x,y)\:dA = \int_a^b\int_{g_1(x)}^{g_2(x)}f(x,y)\:dy\:dx$$

\subsection{Type B Regions}
Left and right boundaries are curves given by $x=h_1(y)$ and $x=h_2(y)$ and bottom and top boundaries are straight lines $y=c$ and $y=d$ respectively.
$$R: c\leq y\leq d,\: h_1(y)\leq x\leq h_2(y)$$
$$\iint_R f(x,y)\:dA = \int_c^d\int_{h_1(y)}^{h_2(y)}f(x,y)\:dx\:dy$$

\subsection{Polar Coordinates}
Circular regions/sectors can be described with polar coordinates $r$ and $\theta$.

In general, a region $R$ in polar coordinates is described by
$$R: a\leq r\leq b,\: \alpha\leq\theta\leq\beta$$

When transforming from Cartesian to polar coordinates, $(x,y)$ is transformed to $(r,\theta)$ where
$$x=r\cos{\theta} \text{ and } y=r\sin{\theta}$$
and $dA$ is changed from $dx\:dy$ to $r\:dr\:d\theta$.

\subsection{Application of Double Integrals}
Suppose $D$ is a solid region under a surface defined by $f(x,y)$ over a plane region $R$.
$$\text{Volume of $D$} = \iint_Rf(x,y)\:dA$$

If $f$ has continuous first partial derivatives on a closed region $R$ of the $xy$-plane, then the area $S$ of that portion of the surface $z=f(x,y)$ that projects onto $R$ is given by
$$S = \iint_R\sqrt{\Big( \frac{\partial z}{\partial x}\Big)^2 + \Big( \frac{\partial z}{\partial y} \Big)^2 + 1}\:dA$$

\section{Line Integrals}
\subsection{Vector Fields}
A vector field on $R$ is a vector function $\mathbf{F}$ that assigns to each point a vector $\mathbf{F}(x,y,z)$.
$$\mathbf{F}(x,y,z) = P(x,y,z)\mathbf{i} + Q(x,y,z)\mathbf{j} + R(x,y,z)\mathbf{k}$$

\subsection{Gradient Fields}
$$\nabla f(x,y,z) = f_x(x,y,z)\mathbf{i}+f_y(x,y,z)\mathbf{j} + f_z(x,y,z)\mathbf{k}$$

\subsection{Conservative Fields}
A vector field $\mathbf{F}$ is called a conservative vector field if it is the gradient of some scalar function $f$ such that $\mathbf{F} = \nabla f$, where $f$ is known as  the potential function for $\mathbf{F}$.

Let $\mathbf{F}(x,y) = P(x,y)\mathbf{i} + Q(x,y)\mathbf{j}$ be a vector field on the $xy$-plane.
$$\text{If }\frac{\partial P}{\partial y} = \frac{\partial Q}{\partial x}\text{ then $\mathbf{F}$ is conservative.}$$

Let $\mathbf{F}(x,y,z) = P(x,y,z)\mathbf{i} + Q(x,y,z)\mathbf{j} + R(x,y,z)\mathbf{k}$ be a vector field on $xyz$-space.
$$\text{If } \frac{\partial P}{\partial y} = \frac{\partial Q}{\partial x}, \frac{\partial P}{\partial z} = \frac{\partial R}{\partial x}, \frac{\partial Q}{\partial z} = \frac{\partial R}{\partial y}, \text{ then $\mathbf{F}$ is conservative.} $$

If $\mathbf{F}$ is a conservative vector field, then $\int_C \mathbf{F}\cdot d\mathbf{r}$ is independent of the path taken.

If $\mathbf{F}$ is a conservative vector field, then $\oint_l \mathbf{F}\cdot d\mathbf{r} = 0$ for any \underline{closed} curve $l$, i.e. a curve with a terminal point that coincides with its initial point.

\subsection{Line Integrals of Scalar Functions}
$$\int_Cf(x,y,z)\:ds = \int_a^b f(x(t),y(t),z(t)) ||\mathbf{r}'(t)|| dt$$
$$=\int_a^b f(x(t),y(t),z(t))\sqrt{\Big(\frac{dx}{dt}\Big)^2 + \Big(\frac{dy}{dt}\Big)^2 + \Big(\frac{dz}{dt}\Big)^2}\: dt$$

\subsection{Line Integrals of Vector Fields}
$$\int_C\mathbf{F}\cdot\: d\mathbf{r} = \int_a^b \mathbf{F}(\mathbf{r}(t))\cdot \mathbf{r}'(t) dt$$

Geometrically, the line integral of $\mathbf{F}$ over $C$ is summing up the tangential components of $\mathbf{F}$ with respect to the arc length of $C$.

$$\int_{-C} f(x,y,z) ds = \int_C f(x,y,z) ds$$
The vector equation of a curve $C$ determines the orientation or direction of $C$.

For some $\mathbf{F}(x,y,z) = P(x,y)\mathbf{i} + Q(x,y)\mathbf{j} + R(x,y)\mathbf{k}$,
$$ \int_C \mathbf{F}\cdot d\mathbf{r} = \int_C P dx + Q dy + R dz = \int_a^b P(\mathbf{r}(t))\frac{dx}{dt} + Q(\mathbf{r}(t))\frac{dy}{dt} dt$$

\subsection{Fundamental Theorem for Line Integrals}
If $f$ is a function of 2 or 3 variables whose gradient $\nabla f$ is continuous,
$$\int_C \nabla f \cdot d\mathbf{r} = f(\mathbf{r}(b))-f(\mathbf{r}(a))$$

\subsection{Green's Theorem}
Let $D$ be a bounded region in the $xy$-plane and $\partial D$ the boundary of $D$. Suppose $P(x,y)$ and $Q(x,y)$ has continuous partial derivatives on $D$. Thus,
$$\oint_{\partial D} P dx + Q dy = \iint_D \Big( \frac{\partial Q}{\partial x} - \frac{\partial P}{\partial y} \Big) dA$$

The orientation of $\partial D$ is such that, as one traverses along the boundary in this direction, the region $D$ is always on the left-hand side, i.e. the positive orientation of the boundary.

\section{Surface Integrals}
A parametric representation of a surface is given by
$$\mathbf{r}(u,v) = x(u,v)\mathbf{i} + y(u,v)\mathbf{j} + z(u,v)\mathbf{k}$$

\subsection{Standard Parametric Representation: Sphere}
For a sphere of radius $a$: $x^2+y^2+z^2=a^2$
$$\mathbf{r}(u,v) = (a\sin{u}\cos{v})\mathbf{i}+(a\sin{u}\sin{v})\mathbf{j} + (a\cos{u})\mathbf{k} $$

When $0\leq u\leq\pi$ and $0\leq v\leq 2\pi$, the representation gives a full sphere.

When $0\leq u\leq\frac{\pi}{2}$ and $0\leq v\leq 2\pi$, the representation gives the upper hemisphere.

\subsection{Standard Parametric Representation: Cylinder}
For a circular cylinder of radius $a$: $x^2+y^2=a^2$
$$\mathbf{r}(u,v) = (a\cos{u})\mathbf{i} + (a\sin{u})\mathbf{j} + v\mathbf{k}$$

Here, $u$ measures the angle from the positive $x$-axis about the $z$-axis while $v$ measures the height from the $xy$-plane along the cylinder.

The same applies for $x^2+z^2=a^2$ (cylinder about $y$-axis)
$$\mathbf{r}(u,v) = (a\cos{u})\mathbf{i} + v\mathbf{j} + (a\sin{u})\mathbf{k}$$

and $y^2+z^2=a^2$ (cylinder about $x$-axis)
$$\mathbf{r}(u,v) = v\mathbf{i} + (a\cos{u})\mathbf{j} + (a\sin{u})\mathbf{k}$$

\subsection{Tangent Planes}
Let $S$ be a surface given by the parametric representation
$$\mathbf{r}(u,v) = x(u,v)\mathbf{i} + y(u,v)\mathbf{j} + z(u,v)\mathbf{k}$$

For some position vector $\mathbf{r}_0 = \mathbf{r}(u_0,v_0)$ at a point $P_0$,

Fixing $v=v_0$ for a resulting curve $C_1$, the tangent vector of the space curve $C_1$ is given by
$$\mathbf{r}_u = \frac{\partial x}{\partial u}(u_0,v_0)\mathbf{i} + \frac{\partial y}{\partial u}(u_0,v_0)\mathbf{j} + \frac{\partial z}{\partial u}(u_0,v_0)\mathbf{k}$$

Fixing $u=u_0$ for a resulting curve $C_2$, the tangent vector of the space curve $C_2$ is given by
$$\mathbf{r}_v = \frac{\partial x}{\partial v}(u_0,v_0)\mathbf{i} + \frac{\partial y}{\partial v}(u_0,v_0)\mathbf{j} + \frac{\partial z}{\partial v}(u_0,v_0)\mathbf{k}$$

Both vectors $\mathbf{r}_u$ and $\mathbf{r}_v$ lie in the tangent plane to $S$ at $P_0$.
Thus, the cross product $\mathbf{r}_u \times \mathbf{r}_v$, assuming it is non-zero, provides a normal vector to the tangent plane to $S$ at $P_0$.

The equation of the tangent plane is described by
$$(\mathbf{r} - \mathbf{r}_0)\cdot(\mathbf{r}_u\times\mathbf{r}_v) = 0$$

\subsection{Surface Integrals of Scalar Functions}
Suppose $f(x,y,z)$ be a function defined on a surface $S$, where we can find $\mathbf{r}(u,v)=x(u,v)\mathbf{i}+y(u,v)\mathbf{j}+z(u,k)\mathbf{k}$ of $S$ over a domain $D$.
$$\iint_S f(x,y,z)\:dS = \iint_D f(\mathbf{r}(u,v))|\mathbf{r}_u\times\mathbf{r}_v|\:dA$$

\subsection{Surface Integrals of Vector Fields}
Let $\mathbf{F}$ be a continuous vector fields defined on a surface $S$ with a unit normal vector $\mathbf{n}$.
The surface integral of $\mathbf{F}$ over $S$ is given as
$$\iint_S \mathbf{F}\cdot\mathbf{n}\:dS \text{ or more simply } \iint_S \mathbf{F}\cdot\: dS$$

This integral is also known as the flux of $\mathbf{F}$ over $S$.

If $S$ is given by a parametric representation $\mathbf{r} = \mathbf{r}(u,v)$ with domain $D$,
$$\iint_S\mathbf{F}\cdot\:dS = \iint_D \mathbf{F}(\mathbf{r}(u,v))\cdot(\mathbf{r}_u\times\mathbf{r}_v)\:dA$$

\subsection{Orientation of Surfaces}
By convention, a curve has a positive orientation if it progresses counter-clockwise.

$$\iint_{-S}\mathbf{F}\cdot\:dS = -\iint_S \mathbf{F}\cdot\:dS$$

\subsubsection{The Little Man Analogy}
Imagine a little man traversing a curve $C$.
His head is pointed in the direction of the normal vector of $C$.
$C$ thus has a positive orientation if the region bounded by the curve is always on his left-hand side.

\subsection{Curl and Divergence}
Let $\mathbf{F}=P\mathbf{i}+Q\mathbf{j}+R\mathbf{k}$ be a vector field in $xyz$-space.
$$\text{curl }{\mathbf{F}} = \Big(\frac{\partial R}{\partial y}-\frac{\partial Q}{\partial z}\Big)\mathbf{i} + \Big(\frac{\partial P}{\partial z}-\frac{\partial R}{\partial x}\Big)\mathbf{j} + \Big(\frac{\partial Q}{\partial x}-\frac{\partial P}{\partial y}\Big)\mathbf{k} $$

The curl of a vector field is also itself a vector field.

$$\text{div }\mathbf{F} = \frac{\partial P}{\partial x} + \frac{\partial Q}{\partial y}+\frac{\partial R}{\partial z}$$

The divergence of a vector field is a scalar function.

\subsection{Del Operator}
$$\nabla = \frac{\partial}{\partial x}\mathbf{i} + \frac{\partial}{\partial y}\mathbf{j} + \frac{\partial}{\partial z}\mathbf{k} $$

The curl and divergence operations can be expressed in terms of the del operator.
$$\text{curl }\mathbf{F} = \nabla\times\mathbf{F} \text{ and } \text{div }\mathbf{F} = \nabla\cdot\mathbf{F}$$

\subsection{Conservative Fields*}
Let $\mathbf{F}$ be a vector field in $xyz$-space.

If $\text{curl }\mathbf{F} = \mathbf{0}$, then $\mathbf{F}$ is a conservative field.
The converse is also true.

\subsection{Stokes' Theorem}
Let $S$ be an oriented, piecewise-smooth surface that is bounded by a closed, piecewise-smooth boundary curve $C$.
Let $\mathbf{F}$ be a vector field whose components have continuous partial derivatives on $S$. Then,
$$\oint_C\mathbf{F}\cdot\:d\mathbf{r} = \iint_S (\text{curl }\mathbf{F})\cdot\:dS$$

Stokes' Theorem can also be expressed as
$$ \oint_C P\:dx+Q\:dy+R\:dz =  $$
$$ \iint_S \Big( \frac{\partial R}{\partial y} - \frac{\partial Q}{\partial z}\Big)\,dy\,dz + \Big( \frac{\partial P}{\partial z} - \frac{\partial R}{\partial x}\Big)\,dz\,dx + \Big( \frac{\partial Q}{\partial x}-\frac{\partial P}{\partial y}\Big)\,dx\,dy $$

The orientation of $C$ must be consistent with that of $S$.

\subsection{Gauss' Theorem}
Let $E$ be a solid region, $S$ be the boundary of $E$ given with outward orientation (where the normal vector on the surface always points away from $E$).
Let $\mathbf{F}$ be a vector field whose component functions have continuous partial derivatives in $E$. Then,
$$\iint_S \mathbf{F}\cdot\:d\mathbf{S} = \iiint_E \text{div }\mathbf{F}\:dV$$

\section{Trigonometric Identities}
\subsection{Pythagorean Identities}
$$\sin^2{u} + \cos^2{u} = 1$$
$$1+\tan^2{u} = \sec^2{u}$$
$$1+\cot^2{u} = \csc^2{u}$$

\subsection{Sum and Difference Formulae}
$$\sin(u\pm v) = \sin{u}\cos{v} \pm \cos{u}\sin{v}$$
$$\cos(u\pm v) = \cos{u}\cos{v} \mp \sin{u}\sin{v}$$
$$\tan(u\pm v) = \frac{\tan{u}\pm\tan{v}}{1\mp\tan{u}\tan{v}}$$

\subsection{Double Angle Formulae}
$$\sin{2u} = 2\sin{u}\cos{u}$$
$$\cos{2u} = \cos^2{u} - \sin^2{u} = 2\cos^2{u} - 1 = 1-2\sin^2{u}$$
$$\tan{2u} = \frac{2\tan{u}}{1-\tan^2{u}}$$

\subsection{Half Angle Identities}
$$\sin^2{u} = \frac{1-\cos{2u}}{2}$$
$$\cos^2{u} = \frac{1+\cos{2u}}{2}$$
$$\tan^2{u} = \frac{1-\cos{2u}}{1+\cos{2u}}$$

\subsection{Sum $\to$ Product Identities}
$$\sin{u} + \sin{v} = 2\sin{\frac{u+v}{2}}\cos{\frac{u-v}{2}}$$
$$\sin{u} - \sin{v} = 2\cos{\frac{u+v}{2}}\sin{\frac{u-v}{2}}$$
$$\cos{u} + \cos{v} = 2\cos{\frac{u+v}{2}}\cos{\frac{u-v}{2}}$$
$$\cos{u} - \cos{v} = -2\sin{\frac{u+v}{2}}\sin{\frac{u-v}{2}}$$

\subsection{Product $\to$ Sum Identities}
$$\sin{u}\sin{v} = \frac{1}{2} [\cos{(u-v)} - \cos{(u+v)}]$$
$$\cos{u}\cos{v} = \frac{1}{2} [\cos{(u-v)} + \cos{(u+v)}]$$
$$\sin{u}\cos{v} = \frac{1}{2} [\sin{(u+v)} + \sin{(u-v)}]$$
$$\cos{u}\sin{v} = \frac{1}{2} [\sin{(u+v)} - \sin{(u-v)}]$$

\subsection{Parity Identities}
$$\sin{(-u)} = -\sin{u}$$
$$\cos{(-u)} = \cos{u}$$
$$\tan{(-u)} = -\tan{u}$$

$$\cot{(-u)} = -\cot{u}$$
$$\csc{(-u)} = -\csc{u}$$
$$\sec{(-u)} = \sec{u}$$

{\hfill \texttt{darren.wee@u.nus.edu}}

{\hfill \texttt{Updated \today.}}
\end{multicols}
\end{document}